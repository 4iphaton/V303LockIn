\section{Diskussion}
\label{sec:Diskussion}

Der Versuch dient dazu sich mit der Funktionsweise eines Lock-In-Verstärkers
vertraut zu machen. Dieser soll ein elektrisch verauschtes Signal reinigen
und verstärken. Hierzu wird ein Referenzsignal von \SI{6.72}{\volt}
genutzt, welches für ein deutliches Ausgangssignal mit dem verrauschten
eingehenden Signal in Phase seien sollte. In Abbildung \ref{pic:0p}
ist das integrierte Signal einer gleichgerichteten Sinusspannung zu erkennen.
Die Spannung steigt immer stärker bis die überlagerten Sinussignale
am Maximum angelangt sind. An diesem hat das Ausgangssignal einen Wendepunkt
und flacht bis zum Ende einer Periode wieder ab, bis es bei der
Nullstelle des Eingehenden Signals gegen ein Maximum konvergiert.
Wäre das eingehnde Signal nicht gleichgerichtet, so würde sich
für das integrierte Signal ein einfacher negativer cosinus ergeben.
Stattdessen wird der negative Cosinus in diesem Signal alle \SI{180}{\degree}
abgebrochen und periodisch aneinandergehangen. Betrachtet man mathematisch die Ableitung
von diesem Konstrukt ergibt sich eine periodische Abfolge des Sinus
auf je einer halben Periode, sprich der gleichgerichtete Sinus.
Die folgenden 4 Abbildungen zeigen, wie die integrierte Überlagerung
von Eingangs- und Referenzspannung für verschieden Phasen aussieht. Bei einer Phase von
\SI{180}{\degree} wäre nun wieder das oben beschriebene zu sehen, allerdings mit
negativem Vorzeichen.

Meistens spielt nur die Amplitude des Signals eine Rolle, da
die eigentliche Form des Signals durch das Gleichrichten und das Überlagern
mit einer sinusförmigen Funktion ohnehin verloren geht. Die Amplitude ist
abhängig von der Phase zwischen Eingangs- und Referenzsignal
sowie dem Verstärkungsfaktor, dem sogenannten Gain. In Abbildung \ref{fig:ohne}
ist nun zunächst die Amplitude des Ausgangssignals, welches durch
ein nicht verauschtes Eingangssignal von \SI{10e-3}{\volt} erzeugt wurde,
mit einem Gain von 2000 in Abhängigkeit der Phase zum Referenzsignal zu sehen.
Hierbei wurden die Messwerte in Form der Gleichung \eqref{eqn:jangibmirdiegl}
gefittet. Der Faktor $a$ ist hierbei die  Amplitude $2U_0/\pi$ der Ausgangsspannung,
welcher nach Einberechnung des Gain sehr nah an den \SI{10e-3}{\volt}
Eingangsspannung liegt, jedoch durch das verwenden eines nicht genormten
Referenzsignals leicht nach oben hin verfälscht ist. Der Faktor $b$
berücksichtigt, dass die Generatoren von Eingangs- und Referenzsignal
ein nicht einstellbares, konstantes Offset in der Phase haben und gleicht
dieses aus. In Abbildung \ref{fig:mit} wird das gleiche Signal vor dem Einspeisen
in den Lock-In-Verstärker mithilfe des Noisegenerators so stark verrauscht,
dass es als solches nicht mehr erkennbar ist. Der Fehler dieses Faktors,
welcher durch das Fitten nach \eqref{eqn:err} in Python berechnet wird
ist kleiner, dies liegt jedoch wahrscheinlich daran, dass wir uns bei
diesem Teil bereits besser in die Messgeräte eingefunden hatten. Die Abweichung
zum eigentlichen Wert ist jedoch wie zu erwarten etwas höher geworden, sowie
der Wert von $a$ an sich, da durch das verrauschen zusätzliche Spannung hinzugefügt
wird. Jedoch liegen der nicht verrauchte Wert von $a=$ \SI{11.84(47)e-3}{\volt}
und der verrauschte Wert von $a=$ \SI{12.77(29)e-3}{\volt} sehr nah beieinander,
wodurch die Funktionsweise des Lock-In-Verstärkers als Siganlfilter verifiziert wird.

Die Funktionsweise des Lock-In-Verstärkers als Verstärker sehr schwacher
elektrischer Signale mit Hintergrundrauschen wird durch ein Experiment
mit einer Leuchtdiode und einem Fotosensor nachgeprüft. Die Intensität
der Diode fällt bekanntermaßen mit dem quadratischen Abstand ab.
In Plot \ref{fig:diode} ist hierbei die zur Intensität proportionale
Spannung an dem Fotosensor in Abhängigkeit der Entfernung aufgetragen,
wobei die Messung durch einen Lock-In-Verstärker untestützt ist.
Der hierzu angefertigte Fit zeigt, dass die durch den Lock-In-Verstärker
hervorgehobenen Signale bis \SI{30}{\centi\meter} Entfernung noch sinnvoll gemessen werden
können. Es muss jedoch beachtet werden, dass die ersten 5 Werte aufgrund von
hoher Messungenauigkeit nicht verwendet werden.
Das verstärkte Signal kann auch aus einer Entfernung von bis zu
\SI{85.5}{\centi\meter} noch gemessen werden. Ohne den Verstärker lässt sich
das Signal auf \SI{10}{\centi\meter} kaum noch erkennen.
Hierdurch zeigt sich, dass der Lock-In-Verstärker ein sehr gutes Hilfsmittel
zum erkennen schwacher verrauschter Signale ist, zumindest sofern man
ihre Frequenz kennt und somit die Referenzspannung sinnvoll einstellen kann.
