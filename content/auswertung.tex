\section{Auswertung}
\label{sec:Auswertung}
\subsection{Signalstärke in Abhängigkeit der Phase}



\begin{table}[H]
  \centering
  \caption{Phasenabhängigkeit mit und ohne Störsignal}
  \label{tab:data1}
  \sisetup{table-format=2.1}
  \begin{tabular}{S[table-format=3.0] S S}
    \toprule
    {$\text{Phase}$}&{$U_\text{out}\text{ ohne Störung}$}&{$U_\text{out}\text{ mit Störung}$} \\
    {$/\text{grad}$}&{$/V$}&{$/V$} \\
    \midrule
    0 &   6.6 &   5.6 \\
   15 &   2.2 &   2.8 \\
   30 &  -3.8 &  -4.2 \\
   45 & -14   & -11.6 \\
   60 & -18   & -19   \\
   75 & -22   & -22.5 \\
   90 & -25   & -24.5 \\
  105 & -23.6 & -25   \\
  120 & -20.8 & -24.5 \\
  135 & -17.6 & -22.5 \\
    \bottomrule
  \end{tabular}
\end{table}
[H]

\begin{figure}[H]
  \centering
  \includegraphics{build/phasen.pdf}
  \caption{Änderung der Phase ohne Störung.}
  \label{fig:ohne}
\end{figure}
\begin{figure}[H]
  \centering
  \includegraphics{build/phasen_mit.pdf}
  \caption{Änderung der Phase mit Störung.}
  \label{fig:mit}
\end{figure}

Regressionsgraphen der Form
\begin{equation}
  U(x)=a\text{cos}(\phi + b)
\end{equation}
Für die Messung ohne Störung ergibt sich:
\begin{gather*}
  a=\SI{23.68(94)}{\volt}\\
  b=\SI{1.32(4)}{}
  \intertext{unter Berücksichtigung des Gain x2000}
  a=\SI{11.84(47)e-3}{\volt}
\end{gather*}
Für die Messung mit Störung ergibt sich:
\begin{gather*}
  a=\SI{25.53(58)}{\volt}\\
  b=\SI{1.27(2)}{}
  \intertext{unter Berücksichtigung des Gain x2000}
  a=\SI{12.77(29)e-3}{\volt}
\end{gather*}
\subsection{Intensität der Diode}

\input{content/tables/diode.tex}[H]
\begin{figure}[H]
  \centering
  \includegraphics{build/diode.pdf}
  \caption{Lichtintensität beim Abstand r.}
  \label{fig:diode}
\end{figure}

Regressionsgraph der Form
\begin{equation}
  U(x)=\frac{a}{r^2}
\end{equation}
Es ergibt sich:
\begin{gather*}
  a=\SI{24.54(24)e-3}{\volt}
  \intertext{unter Berücksichtigung des Gain x40}
  a=\SI{0.613(6)e-3}{\volt}
\end{gather*}
